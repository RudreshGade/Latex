\documentclass{beamer}
\usetheme{Madrid}
\usepackage{amsmath,amssymb,amsfonts}
\usepackage{tikz}
\usepackage{graphicx}
\usepackage{mathrsfs}
\usepackage{hyperref}
\usepackage[backend=bibtex,style=authoryear]{biblatex} % or numeric, alphabetic

\title{Zero-Free Regions for Zeta and L-Functions}
\subtitle{Classical Results and Extensions to Maass Forms}
\author{Rudresh D.\ Gade }

% \institute{Mathematics Department, IISER Pune}
\date{09/10/2025}
\def\T{\mathbb{T}}
\def\F{\mathbb{F}}
\def\A{\mathbb{A}}
\def\H{\mathbb{H}}
\def\Ker{\operatorname{Ker}}
\def\Sym{\operatorname{Sym}}
\def\Im{\operatorname{Im}}
\def\Re{\operatorname{Re}}
\def\deg{\operatorname{deg}}
\def\End{\operatorname{End}}
\def\Ext{\operatorname{Ext}}
\def\Hom{\operatorname{Hom}}
\def\Aut{\operatorname{Aut}}
\def\re{\operatorname{re}}
\def\im{\operatorname{im}}
\def\Gal{\operatorname{Gal}}
\def\id{\operatorname{id}}
%\def\mod{\operatorname{mod}}
\def\dim{\operatorname{dim}}
\def\disc{\operatorname{disc}}
\def\exp{\operatorname{exp}}
\newtheorem{prop}[theorem]{Proposition}


\begin{document}

% Custom Title Page with perfectly centered block
\begin{frame}[plain]
  \vfill

  \begin{center}
    % makebox ensures the beamercolorbox is centered horizontally
    \makebox[\textwidth][c]{%
      \begin{beamercolorbox}[rounded=true,shadow=true,wd=0.75\textwidth,center]{title}
        \centering
        {\LARGE\bfseries Zero-Free Regions for Zeta and L-Functions}\\[0.3cm]
        {\large Classical Results and Extensions to Maass Forms}
      \end{beamercolorbox}%
    }
  \end{center}

  \vfill

  \begin{center}
    {\large Rudresh D.\ Gade}\\[0.2cm]
    {\large Supervisor: Dr. Kaneenika Sinha}\\[0.2cm]
    {\large Expert: Dr. Alia Hamieh}\\[0.2cm]

    Mathematics Department, IISER Pune\\[0.6cm]
    {\large 09/10/2025}
  \end{center}

  \vspace*{0.5cm}
\end{frame}

\begin{frame}{Overview}
\tableofcontents
\end{frame}

\section{Introduction and Motivation}

\begin{frame}{The Riemann Zeta Function}{Introduction and Motivation}
\begin{itemize}
\item The Riemann zeta function is $\zeta(s) = \sum_{n=1}^{\infty} \frac{1}{n^s}$ for $\Re(s) > 1$
\item It has an analytic continuation to entire complex plane except for simple pole at $s = 1$
\item The Riemann Hypothesis: All non-trivial zeros lie on the critical line $\Re(s) = \frac{1}{2}$
\end{itemize}

\end{frame}

\begin{frame}{Classical Zero-Free Regions}{Introduction and Motivation}
\begin{theorem}[De la Vallée Poussin, 1896]
There exists a constant $c > 0$ such that $\zeta(\sigma + it) \neq 0$ for 
$$\sigma \geq 1 - \frac{c}{\log(|t| + 2)}$$
\end{theorem}

\vspace{0.5cm}

\begin{itemize}
\item This gives the first explicit zero-free region
\item Improvements have focused on:
  \begin{enumerate}
  \item Better constants $c$
  \item Improved lower bounds for $|t|$
  \item Extensions to L-functions
  \end{enumerate}
\end{itemize}
\end{frame}

\section{Stechkin's Paper: Zeros of the Riemann Zeta Function}

\begin{frame}{Stechkin's Main Result}
\begin{theorem}[Stechkin, 1970]
The Riemann zeta-function $\zeta(\sigma + it)$ has no zeros in the region
$$\sigma \geq 1 - \frac{1}{R \ln t}$$
for $t \geq T$, where $R = 9.65$ and $T = 12$.
\end{theorem}

\vspace{0.5cm}

\begin{block}{Key Innovation}
Stechkin's approach uses:
\begin{itemize}
\item Logarithmic derivative bounds for the gamma function
\item Careful analysis of auxiliary parameters
\item Non-negative trigonometric polynomials
\end{itemize}
\end{block}
\end{frame}

\begin{frame}{Auxiliary Parameter Technique}{Stechkin's Main Result}
\begin{lemma}[Main Observation]
Suppose $u \geq 0$, $0 < \alpha \leq \frac{1}{2}$, $\sigma > 1$, and $\tau = \frac{1}{2}\left(1 + \sqrt{1 + 4\sigma^2}\right)$. Then:
\begingroup\small
\[
\frac{\sigma - \alpha}{(\sigma - \alpha)^2 + u} + \frac{\sigma - 1 + \alpha}{(\sigma - 1 + \alpha)^2 + u}
\geq \frac{1}{\sqrt{5}}\left(\frac{\tau - \alpha}{(\tau - \alpha)^2 + u} + \frac{\tau - 1 + \alpha}{(\tau - 1 + \alpha)^2 + u}\right)
\]
\endgroup
If $u =0$, then \[
\frac{1}{\sigma - \alpha} \geq \frac{1}{\sqrt{5}}\left(\frac{1}{\tau - \alpha} + \frac{1}{\tau - 1 + \alpha}\right)
\]
\end{lemma}

\vspace{0.3cm}

\begin{block}{Remark}
If $u =0$, then \[
\frac{1}{\sigma - \alpha} \geq \frac{1}{\sqrt{5}}\left(\frac{1}{\tau - \alpha} + \frac{1}{\tau - 1 + \alpha}\right)
\]
\end{block}
\end{frame}

\begin{frame}{The Main Lemma}{Stechkin's Main Result}
\begin{lemma}[Lemma 1.3 - Core Estimate]
Let $1 < \sigma \leq \frac{5}{4}$, $\sigma_1 = \frac{1}{2}\left(1 + \sqrt{1 + 4\sigma^2}\right)$, $s = \sigma + it$, $s_1 = \sigma_1 + it$, $t \geq T \geq 12$. Then:
$$f(s) - xf(s_1) \leq \frac{1-x}{2} \ln t + A_3(T)$$
where $f(s) = \Re\left(-\frac{\zeta'}{\zeta}(s)\right)$ and $x = \frac{1}{\sqrt{5}}$.

If $\zeta(s)$ has a zero $\beta + it$ with $\beta > \frac{1}{2}$, then:
$$f(s) - xf(s_1) \leq \frac{1-x}{2} \ln t - \frac{1}{\sigma - \beta} + A_3(T)$$
\end{lemma}
\end{frame}

\begin{frame}{Trigonometric Polynomials Method}{Stechkin's Main Result}
Stechkin uses non-negative trigonometric polynomials:
$$p_n(\phi) = a_0 + a_1 \cos \phi + \cdots + a_n \cos n\phi$$

with conditions:
\begin{itemize}
\item $p_n(\phi) \geq 0$ for all $\phi$
\item $a_k \geq 0$ for $k = 0, 1, \ldots, n$  
\item $a_0 < a_1$
\end{itemize}

\vspace{0.3cm}

Define: $V(p_n) = \frac{p_n(0) - a_0}{(\sqrt{a_1} - \sqrt{a_0})^2} = \frac{a_1 + \cdots + a_n}{(\sqrt{a_1} - \sqrt{a_0})^2}$

\vspace{0.3cm}

The constant $V = \lim_{n \to \infty, p_n \in P_n} \inf V(p_n)$ determines the zero-free region.



\end{frame}

\begin{frame}{Stechkin's Specific Polynomial}{Stechkin's Main Result}
Stechkin then used the following fact \[
\Re\left(\frac{\zeta'}{\zeta}(s)\right) = \sum_{n=1}^{\infty} \frac{\Lambda(n)}{n^s}\cos(t\ln n) 
\] to get the desired bounds.
    The explicit polynomial used:
$$p(\phi) = (0.28 + \cos \phi)^2(0.94 + \cos \phi)^2 $$.
\end{frame}

\section{McCurley's Paper: L-functions of Dirichlet Characters}

\begin{frame}{McCurley's Extension to L-functions}
% \begin{theorem}[McCurley, 1984]
% Let $\chi_1$ and $\chi_2$ be distinct real primitive characters modulo $k_1$ and $k_2$, respectively, and let $\beta_i$ be a real zero of $L(s, \chi_i)$, $i = 1, 2$. Let
% $$M_1 = \max\{k_1k_2/17, 13\} \quad \text{and} \quad R_1 = \frac{5-\sqrt{5}}{15-10\sqrt{2}}$$
% Then: $\min\{\beta_1, \beta_2\} < 1 - \frac{1}{R_1 \log M_1}$
% \end{theorem}
\begin{theorem}
    Let $M = \max (k, k|t|,10) $ and $R = 9.645908801$. Then $\mathcal{L}_k(s)$ has atmost one zero in the region \[
    \left\{s: \sigma >1-\frac{1}{R\ln M} \right\}
    \]
    The only possible zero in this region is a simple real zero arising from an $L$-
function formed with a real non-principal character modulo $k$ where  $\mathcal{L}_k(s)$  is the product of all Dirichlet $L$ function formed by  characters modulo k.
\end{theorem}
\vspace{0.3cm}

\begin{block}{Significance}
This extends zero-free regions from the Riemann zeta function to Dirichlet L-functions and implying that there is maximum one siegel zero
\end{block}
\end{frame}

\begin{frame}{McCurley's Technical Framework}
Key elements of McCurley's approach:

\begin{itemize}
\item \textbf{Cosine polynomial}: $P(\theta) = 8(0.9126 + \cos \theta)^2(0.2766 + \cos \theta)^2$
\item \textbf{Auxiliary function}: $f(t,\chi) = \Re\left\{\frac{1}{\sqrt{5}}\frac{L'}{L}(\sigma_1 + it, \chi) - \frac{L'}{L}(\sigma + it, \chi)\right\}$
\end{itemize}

\vspace{0.3cm}

\begin{block}{Innovation}
McCurley's method adapts Stechkin's technique to handle:
\begin{enumerate}
\item Multiple L-functions simultaneously
\item Conductor-dependent estimates
\item Character-specific bounds
\end{enumerate}
\end{block}
\end{frame}

\section{Introduction to Maass Forms}

\begin{frame}{What are Maass Forms?}
\begin{definition}[Maass Forms]
A Maass form for a discrete subgroup $\Gamma$ of $\text{SL}_2(\mathbb{R})$ is a function $f: \mathcal{H}^* \to \mathbb{R}$ satisfying:
\begin{enumerate}
\item $f(\gamma z) = \left(\frac{\overline{cz+d}}{|cz+d|}\right)^k f(z)$ for all $\gamma = \begin{pmatrix} a & b \\ c & d \end{pmatrix} \in \Gamma$
\item $f$ is an eigenfunction of $\Delta_k = -y^2\left(\frac{\partial^2}{\partial x^2} + \frac{\partial^2}{\partial y^2}\right) + iky\frac{\partial}{\partial x}$
\item $f(z) \ll y^N$ as $y \to +\infty$ for some positive integer $N$
\end{enumerate}
\end{definition}

\vspace{0.3cm}

\begin{block}{Key Difference}
Unlike classical modular forms, Maass forms are not necessarily holomorphic.
\end{block}
We will be focusing on Maass forms with weight $k=0$.
\end{frame}

\begin{frame}{Fourier Expansion of Maass Forms}
\begin{theorem}[Fourier Expansion]
Let $f$ be a Maass form for $\text{SL}_2(\mathbb{Z})$ with eigenvalue $\lambda = \frac{1}{4} + r^2$. Then:
$$f(z) = \sum_{n \neq 0} \lambda_f(n)\sqrt{y} K_{ir}(2\pi|n|y)e(nx)$$
where:
\begin{itemize}
\item $\lambda_f(n) \in \mathbb{R}$ are eigenvalues of Hecke operators $T_n$
\item $\lambda_f(1) = 1$ (normalization)
\item $K_{ir}$ is the modified K-Bessel function
\end{itemize}
\end{theorem}

\vspace{0.3cm}

\begin{block}{L-function}
The associated L-function  for $f$ is: $L(s,f) = \displaystyle\sum_{n=1}^{\infty} \frac{\lambda_f(n)}{n^s}$
\end{block}
\end{frame}
\begin{frame}{Preliminaries for Maass Forms}
    \begin{itemize}
\item  It has been proven that $L(s,f)$ is convergent for $\Re(s)>1$.
\item The completed $L$ function is \[
\Lambda(s,f) = \pi^{-s} \Gamma\left(\frac{s+\epsilon +ir}{2}\right)\Gamma\left(\frac{s+\epsilon -ir}{2}\right)L(s,f)
\] where $\epsilon = 0 \textit{ or }1$ depending on $f$ being even or odd, respectively and $\lambda$ is the eigen value of $f$ with $\lambda = \frac{1}{4} + r^2$.
\end{itemize}
\end{frame}
\section{Zero-Free Regions for L-functions of Maass Forms}

% \begin{frame}{General Framework for L-functions}
% Consider a Dirichlet series $\varphi(s) = \sum_{n=1}^{\infty} \frac{a_n}{n^s}$ with:
% \begin{itemize}
% \item Non-negative coefficients $a_n$
% \item Convergent for $\Re(s) > 1$
% \item Pole of order $m$ at $s = 1$
% \end{itemize}

% Define the completed L-function:
% $$\Lambda(s,\varphi) = s^m(1-s)^m D^s \prod_{i=1}^l \Gamma\left(\frac{s+c_i}{2}\right) \varphi(s)$$

% where $D > 0$ and $c_i = a_i + b_it$ with $\sigma + a_i > 0$.

% \vspace{0.3cm}

% \begin{block}{Hadamard Factorization}
% $$\Lambda(s,\varphi) = e^{A+Bs} \prod_{\rho} \left(1 - \frac{s}{\rho}\right) e^{s/\rho}$$
% \end{block}
% \end{frame}
\begin{frame}{Preliminaries for Maass Forms}
   Let $f$ be a Hecke eigenform  for $\mathrm{SL}_2(\mathbb{Z})$, normalized so that the first Fourier coefficient is $1$. For each prime $p$, the eigenvalue of $f$ for the Hecke operator $T_p$ is denoted $\lambda_f(p)$.

The \textbf{Satake parameters} of $f$ at $p$ are the pair of (possibly complex) numbers $\alpha_p$ and $\alpha_p^{-1}$ defined by
\[
\lambda_f(p) = \alpha_p + \alpha_p^{-1}, \qquad \alpha_p \alpha_p^{-1} = 1.
\]
Equivalently, the standard $L$-function attached to $f$ has the Euler product
\[
L(s, f) = \prod_p \left(1 - \alpha_p p^{-s}\right)^{-1} \left(1 - \alpha_p^{-1} p^{-s}\right)^{-1}.
\]


\end{frame}
\begin{frame}{Preliminaries for Maass Forms}
\begin{definition}[Symmetric Power $L$ - function]
For each integer $m \geq 1$, the $m$-th symmetric power $L$-function of $f$ is given by
\[
L(s, \operatorname{Sym}^m f) := L(s,f,r_m) = \prod_p \prod_{j=0}^m (1 - \alpha_p^{m - 2j} p^{-s})^{-1}.
\]
This $L$-function encodes the arithmetic of the $m$-th symmetric power representation of the local parameter at $p$.
    
\end{definition} 
   We also define the Dirichlet series for  symmetric power $L$ - functions as 
\[
L(s,\Sym^mf):=\sum_{n=1}^\infty \frac{\lambda_{\Sym^mf}(n)}{n^s} \quad \Re(s)> c
\]

The gamma factors for the symmetric power $L$- function are 
\[
\gamma(s,\Sym^m f )= \prod_{j=0}^m \Gamma(\frac{s-(m-2j)ir}{2}) \quad,\quad  \lambda=\frac{1}{4}+r^2
\]    
\end{frame}
\begin{frame}{Preliminaries for Maass Forms}
We know that the completed L function $\Lambda(s,\Sym^mf)$is entire and has functional equation as \[
\Lambda(s,\Sym^mf) = Q_m^s\gamma(s,\Sym^mf)L(s,\Sym^mf) 
\] 
\[
\Lambda(s,\Sym^mf)= \epsilon_{\Sym^mf}\Lambda(1-s,\Sym^mf)  
\]
with $\epsilon_{\Sym^mf} \in \{\pm 1\} $ and 
\[
Q_i = \pi^{\frac{-(i+1)}{2}}
\].
   We also define the following Dirichlet series as 
\begin{align*}
      L_{(0)}(s,f)&= L(s,\Sym^0f)=\zeta(s)\\
L_{(1)}(s,f)&= L(s,f)\\
L_{(2)}(s,f)&= L(s,\chi_0)L(s,\Sym^2f)\\
L_{(3)}(s,f)&= L(s,\Sym^3f)L(s,f)^2\\
L_{(4)}(s,f)&= L(s,\chi_0)^2L(s,\Sym^2f)^3L(s,\Sym^4f) 
\end{align*}
\end{frame}
\begin{frame}{Preliminaries for Maass Forms}
As in \cite{IPENT}, we consider a polynomial of form $P(\theta) = \gamma(a+\cos \theta)^2(b+\cos\theta)^2 = \displaystyle\sum_{i=0}^4 a_i\cos(i\theta)$ where 
\begin{align*}
    a_0 &= \frac{\gamma}{2}\left( \frac{3}{4} + 4ab+b^2+a^2(1+2b^2)\right)\\
    a_1 &= \frac{\gamma}{2}(a+b)(3+4ab)\\
    a_2 &=\frac{\gamma}{2}(1+a^2+4ab+b^2)\\
    a_3 &= \frac{\gamma}{2}(a+b)\\
    a_4 &= \frac{\gamma}{8}
\end{align*}
For $ n = 0 , 1 , 2 , 3 , 4 $ we define the auxiliary functions $j_{f,n}(\sigma,t)$ as follows: 
\[
j_{f,n}(\sigma,t) = \Re\left( \frac{1}{\sqrt{5}}\frac{L'_{(n)}}{L_{(n)}} (\sigma_1+it,f)-\frac{L'_{(n)}}{L_{(n)}} (\sigma+it,f) \right)
\]
\end{frame}
\begin{frame}{Preliminaries for Maass Forms}
    We find upper bound for $j_{f,n}(\sigma,mt)$ for $ (n,m) = \{ (0,0),(2,0),(4,0),(1,1),(3,1),(2,2),(4,2),(3,3),(4,4)\} $ as \[
    j_{f,n}(\sigma,mt)< j_{f,n}^M(\sigma,mt) +j_{f,n}^E(\sigma,mt)
    \]
\begin{block}{Proposition}
     Let $\sigma>1$ and $t \in \mathbb{R}$, then 
    \begin{multline*}
        0\leq 8a^2b^2j(\sigma,0) + (a_2-4)j_{f,2}(\sigma,0)+3j_{f,4}(\sigma,0) +(a_1-3a_3)j_f(\sigma,t) \\+ 3a_3j_{f,3}(\sigma,t) + (a_2-4)j_{f,2}(\sigma,2t) + 4j_{f,4}(\sigma,2t) + a_3j_{f,3}(\sigma,3t) + j_{f,4}(\sigma,4t) 
    \end{multline*} 
\end{block}

\end{frame}
\begin{frame}{Main Result for Maass Forms}
\begin{theorem}
        Let $f$ be an even Maass form and \[A(t) = \left(\max \left(\left( \frac{\phi^2}{4} + 4(|t|+|r|)^2\right),\left( \left(\frac{\phi+2}{2}\right)^2 +4t^2 \right) \right)\right)\]. Then there exists a constant $c > 0$ such that the $L$-function $L(s,f)$ has no zeros in the region
    
        \[
        \sigma \geq 1 - \begin{cases}
        \frac{1}{10.220267750315\ln A(1)} & : |t| < \frac{0.2}{\ln A(1)}\\
        \frac{1}{9.37747\ln A(1)} & : 1>|t| \geq \frac{0.2}{\ln A(1)}\\
        \frac{1}{10.220267750315\ln A(t)} & : |t| \geq 1
        \end{cases}
        \]
    \end{theorem}

\end{frame}

\begin{frame}{Main Result for Maass Forms}
    We divide our region for $|t|$ in 3 parts:
    \begin{enumerate}
        \item $|t|\geq 1$
        \item $ \frac{\gamma}{\ln \left(\max \left(\left( \frac{\phi^2}{4} + 4(1+|r|)^2\right),\left( \left(\frac{\phi+2}{2}\right)^2 +4 \right) \right)\right)}\leq|t|<1$
        \item $\frac{\gamma}{\ln \left(\max \left(\left( \frac{\phi^2}{4} + 4(1+|r|)^2\right),\left( \left(\frac{\phi+2}{2}\right)^2 +4 \right) \right)\right)}>|t|$
    \end{enumerate}
 where $\gamma $ is chosen later appropriately.
    \\For the first case, using the bounds for $j_{f,n}(\sigma,mt)$ and the above proposition, we the following inequality,
\end{frame}
\begin{frame}{Main Result for Maass Forms}
\begin{multline*}
    \frac{a_1+3a_3}{\sigma-\beta_0  } \leq \frac{8a^2b^2+a_2+2}{\sigma-1} + \Big[ \frac{(a_2-4)\kappa}{2} + \frac{15\kappa}{2} + \frac{(a_1-3a_3)\kappa}{2}  \\+6a_3\kappa  + \frac{5(a_2-4)\kappa}{4} +18\kappa+  2a_3\kappa + \kappa\\ + \frac{7\kappa}{2} \Big]  \ln \left(\max \left(\left( \frac{\phi^2}{4} + 4(|t|+|r|)^2\right),\left( \left(\frac{\phi+2}{2}\right)^2 +4t^2 \right) \right)\right) \\  -a^2b^28(0.601655) -(a_2-4)(0.8388)-3(2.23195)\\ -(a_1-3a_3)(0.08) -3a_3(0.320018)-(a_2-4)(0.14654)\\ - 4(0.0019669) +a_3(1.927202)-0.073063
\end{multline*}   
\end{frame}

\begin{frame}{Main Result for Maass Forms}
    For $(a,b) = (1.5315,0.374949)$, we get \begin{multline*}
    K(a,b) =   -a^2b^28(0.601655) -(a_2-4)(0.8388)-3(2.23195)\\ -(a_1-3a_3)(0.08) -3a_3(0.320018)-(a_2-4)(0.14654) - 4(0.0019669) \\+a_3(1.927202)-0.073063  =  -21.2415181449608<0
\end{multline*}
We choose $$\sigma = 1+\frac{x}{\ln \left(\max \left(\left( \frac{\phi^2}{4} + 4(|t|+|r|)^2\right),\left( \left(\frac{\phi+2}{2}\right)^2 +4t^2 \right) \right)\right)}$$ where $x$ is chosen later appropriately. 
With some analysis, we get \[
\beta_0 \leq 1-\frac{1}{10.220267750315\ln \left(\max \left(\left( \frac{\phi^2}{4} + 4(|t|+|r|)^2\right),\left( \left(\frac{\phi+2}{2}\right)^2 +4t^2 \right) \right)\right) }
\]
\end{frame}
\begin{frame}{Main Result for Maass Forms}
    For the second part of $|t|$, we use the following lemma.
    \begin{block}{Lemma}
            For $\sigma>1$ we have\[
    0\leq j_{f,0}(\sigma,0) -2j_{f,1}(\sigma,0) +j_{f,2}(\sigma,0) 
    \]
    \end{block}
Using the bounds, we get \begin{multline*}
    0 \leq \frac{2}{\sigma-1}-\frac{4(\sigma-\beta_0)}{(\sigma-\beta_0)^2+t_0^2}+\\ \frac{\kappa}{2}\ln\left(\max \left(\left( \frac{\phi^2}{4} + 4(1+|r|)^2\right),\left( \left(\frac{\phi+2}{2}\right)^2 +4 \right) \right)\right) 
\end{multline*}
\end{frame}
\begin{frame}{Main Result for Maass Forms}
    where $t_0 = \frac{\gamma}{\ln\left(\max \left(\left( \frac{\phi^2}{4} + 4(1+|r|)^2\right),\left( \left(\frac{\phi+2}{2}\right)^2 +4 \right) \right)\right)}$.\\
    We set $$\sigma = 1+\frac{r_1}{\ln\left(\max \left(\left( \frac{\phi^2}{4} + 4(1+|r|)^2\right),\left( \left(\frac{\phi+2}{2}\right)^2 +4 \right) \right)\right) }$$ and $$\beta_0 = 1-\frac{c}{\ln\left(\max \left(\left( \frac{\phi^2}{4} + 4(1+|r|)^2\right),\left( \left(\frac{\phi+2}{2}\right)^2 +4 \right) \right)\right)   }$$ for some $r_1,c >0$
and get \[
\frac{4(r_1+c)}{(r_1+c)^2+\gamma^2} < \frac{4+kr_1}{2r_1}
\]

\end{frame}
\begin{frame}{Main Result for Maass Forms}
    We solve for $c>0$ and choosing $\gamma = 0.2$ and $r_1 = 1$, we get 
\[
c\geq \frac{1}{10.220267750315 }
\]
    Finally for the last part, we again use the same proposition as in part 1 and using the bounds we get,
    \begin{multline*}
    0 < \frac{8a^2b^2+a_2+2}{\sigma-1} - \frac{a_1+3a_3}{\sigma-\beta_0} - \frac{2a_3(\sigma-\beta_0)}{(\sigma-\beta_0)^2+4t^2} +\frac{(a_2+4)(\sigma-1)}{(\sigma-1)^2 + 4t^2}\\ + \frac{2(\sigma-1)}{(\sigma-1)^2 + 16t^2}+ \left[ 23+\frac{a_1}{2}+\frac{7a_2}{4} + \frac{13a_3}{2}\right]\\\kappa\ln\left(\max \left(\left( \frac{\phi^2}{4} + 4(1+|r|)^2\right),\left( \left(\frac{\phi+2}{2}\right)^2 +4 \right) \right)\right)  \\ -8a^2b^2(0.601655) -(a_2-4)(0.8388)-3(2.23195) -(a_1-3a_3)(0.08)\\ -3a_3(0.320018)-(a_2-4)(0.189189) - 4(0.08725) +a_3(1.927202)-0.08725
\end{multline*}

\end{frame}
\begin{frame}{Main Result for Maass Forms}
    Let 
\begin{multline*}
    D(a,b) = -8a^2b^2(0.601655) -(a_2-4)(0.8388)-3(2.23195)\\ -(a_1-3a_3)(0.08) -3a_3(0.320018)-(a_2-4)(0.189189) - 4(0.08725) \\+a_3(1.927202)-0.08725
\end{multline*}
Now $D(1.5315,0.374949) = -51.8056996464467< 0 $.\\
    Now we set  \[
\sigma = 1+\frac{r_2}{\ln\left(\max \left(\left( \frac{\phi^2}{4} + 4(1+|r|)^2\right),\left( \left(\frac{\phi+2}{2}\right)^2 +4 \right) \right)\right) }
\] and we already have 
\[
\beta_0 = 1-\frac{c}{\ln\left(\max \left(\left( \frac{\phi^2}{4} + 4(1+|r|)^2\right),\left( \left(\frac{\phi+2}{2}\right)^2 +4 \right) \right)\right) }
\]
\end{frame}
\begin{frame}{Main Result for Maass Forms}
  Simplifying the above inequality and removing the $\log$ term, we get 
\[
 \frac{A}{r_2+c} \leq \frac{B}{r_2} + C + \frac{a_2+4}{r_2^2+4\gamma^2} - \frac{2a_3(r_2+c)}{(r_2+c)^2 +4\gamma^2} +\frac{2r_2}{r_2^2+16\gamma^2} 
\]
where $A = a_1+3a_3$, $B = 8a^2b^2+a_2+2$, $C = \left[ 23+\frac{a_1}{2}+\frac{7a_2}{4} + \frac{13a_3}{2}\right]\kappa$.\\
For $\delta$ chosen such that \begin{equation} \label{58}
    \frac{a_2+4}{r_2^2+4\gamma^2} - \frac{2a_3(r_2+c)}{(r_2+c)^2 +4\gamma^2} +\frac{2r_2}{r_2^2+16\gamma^2} -\delta <0
\end{equation}
    Solving for $c>0$ and choosing $\delta = 8.26247$ we get $c= \frac{1}{9.37747}$ 
%     \[
% c>\frac{(A-B)r_2-Cr_2^2-\delta r_2^2}{B+Cr_2+\delta r_2}
% \]
% \begin{multline*}
%     \frac{(a_2+4)(\sqrt{AB}-B)(C+\delta)}{(\sqrt{AB}-B)^2 + 4\gamma^2(C+\delta)^2} + \frac{(2)(\sqrt{AB}-B)(C+\delta)}{(\sqrt{AB}-B)^2 + 16\gamma^2(C+\delta)^2} \\- \frac{(2a_3)(A-\sqrt{AB})(C+\delta)}{(A-\sqrt{AB})^2 + 4\gamma^2(C+\delta)^2} -\delta <0
% \end{multline*}

This is satisfied if we take $\delta = 8.26247$ and get $c = \frac{1}{9.37747}$


\end{frame}
\begin{frame}{Final result for Maass forms}
    Thus combining all the three parts, we get the final result as 
    \begin{theorem}
        Let $f$ be an even Maass form and \[A(t) = \left(\max \left(\left( \frac{\phi^2}{4} + 4(|t|+|r|)^2\right),\left( \left(\frac{\phi+2}{2}\right)^2 +4t^2 \right) \right)\right)\]. Then there exists a constant $c > 0$ such that the $L$-function $L(s,f)$ has no zeros in the region
    
        \[
        \sigma \geq 1 - \begin{cases}
        \frac{1}{10.220267750315\ln A(1)} & : |t| < \frac{0.2}{\ln A(1)}\\
        \frac{1}{9.37747\ln A(1)} & : 1>|t| \geq \frac{0.2}{\ln A(1)}\\
        \frac{1}{10.220267750315\ln A(t)} & : |t| \geq 1
        \end{cases}
        \].
    \end{theorem}
\begin{block}{Remark}
    This result is only true for even Maass forms with weight $k = 0$, as the gamma factors of completed $L$ functions are different for odd Maass forms, and we get a slightly different result.
\end{block}
\end{frame}
\section{Comparison and Applications}

% \begin{frame}{Comparison of Methods}
% \begin{block}{Common Features}
% \begin{itemize}
% \item Logarithmic derivative method
% \item Auxiliary parameter technique
% \item Trigonometric polynomial optimization
% \item Gamma function estimates
% \end{itemize}
% \end{block}
% \end{frame}

\begin{frame}{Applications and Future Directions}
\textbf{Applications of Zero-Free Regions:}
\begin{itemize}
\item Prime number theorem error terms
\item Bounds for coefficients of L-functions
\item Distribution of eigenvalues
\end{itemize}

\vspace{0.5cm}

\textbf{Open Problems:}
\begin{itemize}
\item Extend to odd Maass forms
\item Extension to higher level
\item Higher rank groups $(GL(n))$

\end{itemize}
\end{frame}
\begin{frame}
\begin{center}
\Large \textbf{Thank you for your attention!}
\end{center}
\end{frame}

\begin{frame}{References}
\footnotesize
\begin{itemize}
\item Stechkin, S.B.: The zeros of the Riemann zeta-function. Mat. Zametki 8 (1970), 419-429
\item McCurley, K.S.: Explicit zero-free regions for Dirichlet L-functions. J. Number Theory 19 (1984), 7-32
\item Creech, S., Hamieh, A., Khunger, S., Sinha, K., Streipel, J., Tsang, K.M.: Explicit zero-free regions for automorphic L-functions. arXiv preprint arXiv:2402.07602 (2024)
\end{itemize}
\end{frame}
\end{document}
